
乍一看,在本章开头就谈论软件质量似乎有些奇怪。
这似乎与操作系统关系不大。但事实并非如此,我们可以在这里学到一些宝贵的经验。

如果让你定义“高质量”软件的含义,你会怎么说?不妨看看以下几点:

\begin{itemize}
    \item 它应该正确地完成其工作(“功能”正确性)。
    \item 它应该在正确的时间内完成其工作(“时间”正确性)。
    \item 其行为应该是可预测的。
    \item 其行为应该是一致的。
    \item 代码不应难以维护(低复杂度)。
    \item 代码的正确性可以被分析(静态分析)。
    \item 代码的行为可以被分析(覆盖率分析)。
    \item 运行时性能应该是可预测的。
    \item 内存需求应该是可预测的。
    \item 如果需要,代码可以被证明符合相关标准。
\end{itemize}

当然,你可以根据自己的特定需求扩展这个列表。

现在,考虑将这些原则应用于下面这个小型、相对简单的实时系统(图1.1)。

% 这里有个图

这里的要求是通过改变液体的流速来控制其温度,它是通过以下步骤完成的:

\begin{itemize}
    \item 使用温度传感器测量液体温度。
    \item 将其与期望的温度值进行比较。
    \item 生成一个控制信号,以设置控制冷却剂流量的执行器的位置。
\end{itemize}

软件必须执行:

\begin{itemize}
    \item 数据采集。
    \item 信号线性化和缩放。
    \item 控制计算。
    \item 执行器驱动。
\end{itemize}

然而,这是一个核反应堆控制系统中的安全完整性等级4(SIL4)的子系统。因此,禁止使用中断。

没有唯一的代码解决方案,但它肯定会是以下形式(代码清单1.1)。

% 这里是代码

我们所拥有的是一个“应用级”代码的例子,这里它被构造成一个单一的顺序程序单元。
还要注意,低层级的细节对我们是隐藏的。

总的来说,低层级操作涉及系统硬件和相关活动。
即使使用高级语言,程序员也必须对机器硬件和功能有专业的知识。
这也凸显了传统微处理器编程的一个问题:
要实现好的设计,需要硬件/软件两方面的专业知识。
即使对于这个简单的例子,程序员也需要相当程度的硬件和软件技能。