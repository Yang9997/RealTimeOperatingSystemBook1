% !TeX root = ../../main.tex

在大型计算机领域,操作系统(OS)已经伴随我们相当长的时间了。
实际上,最初的操作系统可以追溯到20世纪50年代。
到了60年代,操作系统取得了重大进展;
到70年代中期,其概念、结构、功能和接口已经基本确立。

微型计算机大约在1970年出现。
从逻辑上来看,操作系统似乎应当能够迅速应用于基于微处理器的系统中。
然而,到80年代中期,真正采用正式设计的实时操作系统(RTOS)的微处理器实现仍然很少。
确实,CP/M于1975年发布,并后来被Intel制作成硅片产品,
但它对实时领域影响不大,其自然应用领域是桌面计算机。

RTOS的应用受两个因素影响:
一方面是机器本身的限制,另一方面是围绕微型计算机的设计文化。
早期的微型计算机在计算能力、运行速度和内存容量上都非常有限。
想要在这种基础上引入操作系统结构是非常困难的。
此外,当时大多数从事嵌入式系统编程的人,对操作系统几乎没有背景知识。

如今情况则大不相同。现代嵌入式设计的核心工作马是16/32位复杂微控制器。
这些器件低成本、高性能,并且集成了大量片上存储和外设功能。
此外,市场上有非常多的商业 RTOS 可供选择。
然而,仅仅因为你能够做某件事,并不意味着你就应该去做。
那么,为什么在下一个设计中你应该选择使用RTOS呢?

在回答这个问题之前,我们需要先考虑一个更基础的问题:
在嵌入式系统中,我们到底应该如何开展软件设计?
这正是本章的关键点。
本章为实用的设计技术奠定基础,并展示了RTOS在其中的具体作用。