% !TeX root = ../../main.tex

请大家——无论有经验与否——务必阅读第1章。不仅要读,还要真正理解其中的内容。
因为如果你没有真正理解这里讨论的问题,你将很难设计出好方案。

第2章到第6章讲述了该领域的基础知识。
它们不仅展示了多任务设计是如何实现的,还解释了为什么要以某种方式去实现。
这里的目标读者是那些刚接触实时嵌入式系统任务设计与实现的人。
但需要强调的是,本节主要将工作放在单处理器单元的背景下。
第7章和第8章则将视角拓宽到多处理器和分布式系统
(需要说明的是,这两者之间的界限有时并不十分明确)。

第9章本质上偏理论,但带有实用倾向,从更广的角度讨论任务调度技术。
之所以把这一章放在后面,是为了让读者更容易吸收内容。
只要读者熟悉本领域的基本概念,这一章应该相对容易理解。

第10章到第12章侧重于该领域的实际操作。
如果你刚接触RTOS领域,第10章会帮助你扎实理解不同操作系统结构之间的差异,
这在你选择第一个RTOS时尤其有用。
相比之下,第11章和第12章的内容在你搭建系统之后会特别有帮助。
本质上,它们关注的是软件在运行时的行为、质量和可靠性。