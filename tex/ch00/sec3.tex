
本书主要介绍该领域的基础知识。简单来说,它试图解答以下几个主要问题:

\begin{enumerate}
    \item 什么是实时操作系统(RTOS)?
    \item 为什么在设计中应该使用RTOS?
    \item 使用RTOS会有不利之处吗?
    \item 嵌入式实时操作系统的组成模块有哪些?
    \item 现代嵌入式系统可能采用单处理器、多处理器或多台计算机。我们应如何在这些不同平台上使用RTOS?
    \item 如何评估RTOS的性能,并在必要时进行改进?
    \item 如何对基于RTOS的设计进行调试?
\end{enumerate}

目录部分会更详细地展示本书内容安排;此外,每章开头都会明确说明本章的目标。我建议你先浏览这些内容,以对本书整体范围和目的有一个清晰的了解。
原书已重新命名为《实时操作系统第1册——理论篇》。之所以这样,是因为本书有一部配套书籍《实时操作系统第2册——实践篇》。
该配套书包含了一系列练习(当然,你可以选择是否完成),以帮助你更好地理解相关主题。
这些练习主要涉及第1至第5章中讲解的核心内容。我建议你在学习理论材料的同时进行实践练习,这样可以让你更好地应对实际基于 RTOS 的设计问题。